\documentclass{article}
\usepackage[ngerman]{babel} 
\usepackage{hyperref}
\usepackage{amsmath}
\usepackage{amssymb}

\usepackage{xcolor} % Für Farben
\usepackage{tcolorbox} % Für Kästen

\usepackage{graphicx} % Required for inserting images

\title{Abgabe 1 Computergestützte Methoden}
\author{Gruppe 94, Melissa Aygül,Tamara Markovic}
\date{14.11.2024}

\begin{document}

\maketitle
\tableofcontents

\newpage

\section{Der Zentrale Grenzwertsatz}
Der Zentrale Grenzwertsatz (ZGS) ist ein fundamentales Resultat der Wahr-
scheinlichkeitstheorie, das die Verteilung von Summen unabhängiger, identisch verteilter (\textit{i.i.d}) Zufallsvariablen (ZV) beschreibt. Er besagt, dass unter bestimmten Voraussetzungen die Summe einer großen Anzahl solcher ZV annähernd normalverteilt ist, unabhängig von der Verteilung der einzelnen ZV. Dies ist besonders nützlich, da die Normalverteilung gut untersucht und mathematisch handhabbar ist.

\subsection{Aussage}
Sei $X_1$,$X_2$,...,$X_n$ eine Folge von \textit{i.i.d.} ZV mit dem Erwartungswert $\mu = \mathbb{E}(X_i)$ und der Varianz $\sigma^2 = \text{Var}(X_i)$, wobei $0 < \sigma^2 < \infty$ gelte. Dann konvergiert
die standardisierte Summe $Z_n$ dieser ZV für $n \to \infty$
in Verteilung gegen eine Standardnormalverteilung:
\footnote{Der zentrale Grenzwertsatz hat verschiedene Verallgemeinerungen. Eine davon ist der 
\textit{Lindeberg-Feller-Zentrale-Grenzwertsatz} \cite[Seite 328]{Klenke}, der schwächere Bedingungen an die Unabhängigkeit 
und die identische Verteilung der ZV stellt.}

\begin{equation}
Z_n = \frac{\sum_{i=1}^n X_i - n\mu}{\sigma\sqrt{n}} \overset{d}{\to} \mathcal{N}(0, 1).
\label{eq1}
\end{equation}
Das bedeutet, dass für große $n$ die Summe der ZV näherungsweise normalverteilt ist 
mit Erwartungswert $n\mu$ und Varianz $n\sigma^2$:

\begin{equation}
\sum_{i=1}^n X_i \sim \mathcal{N}(n\mu, n\sigma^2)
\label{eq2}
\end{equation}

\subsection{Erklärung der Standardisierung}
Um die Summe der ZV in eine Standardnormalverteilung zu transformieren, subtrahiert man den Erwartungswert $n\mu$ und teilt durch die Standardabweichung $\sigma\sqrt{n}$. Dies führt zu der obigen Formel \eqref{eq1} .Die Darstellung \eqref{eq2} ist für
$n \to \infty$ nicht wohldefiniert.



\subsection{Anwendungen}

Der ZGS wird in vielen Bereichen der Statistik und der Wahrscheinlichkeitstheorie angewendet. Typische Beispiele sind:

\begin{itemize}
    \item \textit{(ergänzen Sie hier einen Anwendungsfall für den ZGS)}
    \item \textit{(ergänzen Sie hier einen weiteren Anwendungsfall)}
\end{itemize}






\newpage
\section{Bearbeitung zu Aufgabe 1}
\subsection{Thema Datenverarbeitung}
\includegraphics[width = \textwidth]{Exel.png}
\begin{enumerate}
 \item Die Tabelle enthält eine Sammlung von Daten zur Fahrradstation „W 87 St and West End Ave“, die unserer Gruppe 94 zugeordnet ist. Diese Daten umfassen tägliche Wetterinformationen und einige zusätzliche Details, die zur Analyse der Fahrradnutzung in Abhängigkeit von Wetterbedingungen verwendet werden können.
Das Datum wird in der Spalte „date“ angegeben, zusammen mit Tag des Jahres („day\_of\_year“) und Wochentag („day\_of\_week“). Außerdem wird der Monat des jeweiligen Datums in der Spalte „month\_of\_year“ festgehalten.Wichtige Wetterdaten umfassen die Niederschlagsmenge („precipitation“) und die Windgeschwindigkeit („windspeed“), die Hinweise darauf geben können, ob und wie Wetterfaktoren wie Regen oder Wind die Nutzung der Fahrradstation beeinflussen. Temperaturdaten werden durch drei Spalten dokumentiert: minimale Temperatur („min\_temperature“), durchschnittliche Temperatur („average\_temperature“) und maximale Temperatur („max\_temperature“), alle in Fahrenheit. 
Eine weitere wichtige Spalte ist „count“, die die Zahl der entliehenen Fahrräder angibt. Durch diese Zählung kann analysiert werden, ob und in welchem Umfang die Fahrradnutzung von den Wetterbedingungen beeinflusst wird.
\item Um den Datensatz in Excel zu importieren, haben wir zunächst Excel geöffnet und den Pfad „Daten“ gewählt. Dort haben wir dann auf „Daten abrufen und transformieren“ geklickt. Anschließend wählten wir „Daten abrufen“ und die Option „Aus Datei“ und „Aus CSV“ aus, um den bereitgestellten Datensatz zu öffnen.
Nachdem der Datensatz importiert war, erhielten wir eine erste Übersicht der kompletten Daten und starteten den Transformationsprozess, um die Daten in das gewünschte Format zu bringen. Nach der Transformation lagen uns alle relevanten Daten vor.
Um nur die für uns wichtigen Informationen zu sehen, haben wir in der Tabelle Filter gesetzt. Besonders haben wir in der Spalte „group“ nach unserer zugewiesenen Gruppe gesucht, in diesem Fall 94. Durch die Eingabe der Zahl im Suchfeld konnten wir die Tabelle auf die für uns wichtigen Datensätze einschränken. So entstand eine übersichtliche und aussagekräftige Tabelle, die nur die benötigten Informationen enthielt.
\item Um die höchste mittlere Temperatur zu ermitteln haben wir die Werte in der Spalte „average _temperature“ der Größe nach (absteigend) sortiert. 
Zusätzlich zur Temperatur in Fahrenheit wurde eine Spalte „Mittlere Temperatur Celsius“ eingefügt, die die Durchschnittstemperatur in Celsius berechnet. Die Umrechnung erfolgt anhand der Formel ((\_temperature - 32) * 5/9). In Fällen, in denen die ursprüngliche Durchschnittstemperatur fehlt („NA“), zeigt diese Spalte Fehler an. Die höchste „average\_temperature“ wurde am 28.07.23 mit 83 Grad Fahrenheit gemessen. In diesem Fall berechnet man (86-32)*(5/9) und erhält das gerundete Ergebnis 28,33 Grad Celsius.
\end{enumerate}
\newpage


\bibliographystyle{plain}
\bibliography{references}

\url{https://github.com/Melissa-ayl/Comet-Abgabe }
hallo
\end{document}


